\chapter{Requirement Analysis}
\section{User Side Requirements}
\begin{enumerate}
 \item User should have an android mobile device running OS version 2.1 or above.
 \item User should have Android Application installed within the mobile.
 \item User must be able to connect the device using USB cable
 \item User must be able to connect device using Wi-Fi.
 \item User must have internet plan enabled on his device or use Wi-Fi connection for
online synchronization.
 \item Interface must be user friendly and attractive.
\end{enumerate}

\section{System Side Requirements}
\begin{enumerate}
 \item System must have Android based drivers for USB and Wi-Fi adapters.
 \item The server side application must be installed on the standard PC.
 \item System must be connected to the internet for online synchronization.
 \item Interface must be user friendly and attractive.
\end{enumerate}

\section{Hardware and Software Requirements}
\begin{enumerate}
 \item Android Based Mobile with minimum of:
 \begin{enumerate}[a. ]
  \item 128KB non-volatile memory to run Mobile Information Device (MID).
  \item 8KB of non-volatile memory for storage of persistent application data.
  \item 2KB of volatile memory to run JVM
 \end{enumerate}
 \item Device should have Wi-Fi connectivity.
 \item 600mhz processor
 \item 150 MB of RAM
\end{enumerate}

\section{Analysis of Data}
\hspace*{0.82cm}All the information collected on Android operating system is obtained from
developer.android.com. Since Android is an open source, many of the codes are readily
available for application development, which has rapidly added to the success of Android
Systems. It gives budding engineers a platform for learning and development. The data
mentioned on the website developer.android.com explains the changes to be made in the
manifestation file for installing a new application. Moreover it explains the procedure of
connecting a USB device to Android supported phone. The two modes of android device viz.
Host mode and Accessory mode are mentioned on this website. These modes can be used for
offline synchronization of the data. The data obtained from this website proves to be
tremendously helpful in the android application development for the synchronization process.