\chapter{System Design}
\section{Goals of Design}
\begin{enumerate}
 \item The project design should help us to visualize the system as it is or want it to be.
 \item It should permit us to specify the structure or behavior of the system.
 \item To give us template that guides us in the construction of the system.
 \item To document the decision we have made.
\end{enumerate}

\section{UML Diagrams}
\hspace*{0.82cm}The OMG's Unified Modeling Language (UML) helps specify, visualize, and
document models of software systems, including their structure and design, in a way that
meets all of these requirements. Using any one of the large number of UML-based tools on
the market, one can analyze future application's requirements and design a solution that meets
them, representing the results using UML's twelve standard diagram types. One can model
just about any type of application, running on any type and combination of hardware,
operating system, programming language, and network, in UML. Its flexibility helps model
distributed applications that use just about any middleware on the market. Built upon the
MOF metamodel which defines class and operation as fundamental concepts, it's a natural fit
for object-oriented languages and environments such as C++, Java and the recent C\#, but one
can use it to model non-OO applications as well in, for example, Fortran, VB, or COBOL.\\[0.5cm]
\hspace*{0.82cm}Architects design buildings. Builders use the designs to create buildings. The more
complicated the building, the more critical the communication between architect and builder.
Blueprints are the standard graphical language that both architects and builders must learn as
part of their trade.\\[0.5cm]
\hspace*{0.82cm}Writing software is not unlike constructing a building. The more complicated the
underlying system, the more critical the communication among everyone involved in creating
and deploying the software. In the past decade, the UML has emerged as the software
blueprint language for analysts, designers, and programmers alike. It is now part of the
software trade. The UML gives everyone from business analyst to designer to programmer a
common vocabulary to talk about software design.\\[0.5cm]
\hspace*{0.82cm}The UML is applicable to object-oriented problem solving. Anyone interested in
learning UML must be familiar with the underlying tenet of object-oriented problem solving 
it all begins with the construction of a model. A model is an abstraction of the underlying
problem. The domain is the actual world from which the problem comes.\\[0.5cm]
\hspace*{0.82cm}Models consist of objects that interact by sending each other message. Think of an
object as "alive." Objects have things they know (attributes) and things they can do
(behaviors or operations). The values of an object's attributes determine its state.\\[0.5cm]
\hspace*{0.82cm}Classes are the "blueprints" for objects. A class wraps attributes (data) and behaviors
(methods or functions) into a single distinct entity. Objects are instances of classes.\\[0.5cm]
At the center of the UML are its nine kinds of modeling diagrams, which we describe here.
\begin{itemize}
 \item Use case diagrams
 \item Class diagrams
 \item Object diagrams
 \item Sequence diagrams
 \item Collaboration diagrams
 \item Statechart diagrams
 \item Activity diagrams
 \item Component diagrams
 \item Deployment diagrams
\end{itemize}
To model our system we have used the following diagrams:

