\chapter{System-Testing}
software testing: 
Software testing is the process used to assess the quality of computer software. Software testing is an empirical technical
investigation conducted to provide stakeholders with information about the quality of the product or service under test, with respect to the context in which it is 
intended to operate. this includes, but is not limited to, the process of executing a program or application with the intent of finding software bugs. 

there are many approaches to software testing. Reviews, walkthroughs or inspection are considered as static testing, whereas actually running the program with a given set of test cases in a given
development stage is referred to as dynamic testing. 

Software testing is used in asscoiation with verification and validation: 
Verification: Have we built the software right(i.e., does it match the specification)?

Validation: have we built the right software (i.e., is this what the customer wants?)

software testing can be done by software testier. until the 1950s the term software tester was used generally, but later it was also seen as 
a separate profession. Regarding the periods and the different goals in software testing there have been established different roles:
test lead/manager , tester, test designer, test automater/automation developer, and test administrator.

test plan:
it is a systematic approach to testing a system such as a machine or software. The plan typically contains a detailed understanding of what the eventual
workflow will be. 

Test bed:
It is a platform for experimentationf or large development projects. Test beds allow for rigorous, transparent and replicable testing
of scientific theories, computational tools, and other new technologies. 

Test environment: 
In software, the hardware and software requirements are known as the test bed. This is also known as the test environment. 

scenario testing:
Scenario testing is a software testing acitivity that uses scenario tests, or simply scenarios, which are based on a hypothetical story to help a person 
think through a complex problem or system. they can be as simple as a diagram foor a testing environment or they could be a description written in prose. 

Test case:
Test case in software engineering is a set of conditions or variables under whicha tester will determine if a requirement or use case upon an applicaiton is aprtially or full satisfied. 
it may take many test cases to determine that a requirement fully satisfied. Test cases are often incorrectly referred
to as test scripts. Test scripts are lines of code used mainly in automation tools. written test cases are usually collected into test suites. 


Unit testing:

Unit testing tests the minimal software component or mdoule. each unit (basic component) of the software is tested to verify that the detailed design for the unit has been correctly implemented. 

In an orject oriented environment, this is usually at the class level, and the minimal unit tests include the constructors and destructors. 

Integration testing exposes defects in the interfaces and interaction between integrated components (modules. ) Progressively larger groups of testied software components corresoonding to elemnts of the architectural design are integrated and 
testing until the software works as a system. 

System testing tests a completely integrated system to verify that it meets its requirements. 

system integration testing verifies that a system is integrated to any external or third party systems defined in the system 
requirements. 

Black box testing:

bloack box testing treates the software as a black box without any understanding of internal behavior. it aims to test the functionality according
to the requirements. thus the tester intputs data and only sees the output from the test object. 

THis level of testing usually requires through test cases to be provided to the tester who then can simply verify that for a given input, the output value (or behaviour),
is the same as the expected value specified in the test case. Black box testing methods include: equivalence partitioning , boundary value analysis, all - pairs testing, 
fuzz testing , model-based testing, traceability matrix etc. 


White box testing: 

white box testing, however is when the tester has access to the internal data structures, code, and algorithms. White box testing methods include
creating tests to satisfy some code coverage criterial . for Example, the test designer can create tests to cause all statements in the program to be executed at least once. 

Other examples of white box testing are mutation testing and fault injection emthods. 
White box testing includes all static testing. 


Alpha Testing: alpha testing is simulated or actual operational testing by potential users/customers or an independent test team at the developer's site. 
Alpha testing is often emplyed for off-the-shelf software as a form of internal acceptance testing. before the software goes into beta testing. 

Beta testing: 
beta testing comes after alpha testing. Version of the software , known as beta versions , are released to a limited audience outside of the programming team. The software is 
released to groups of people so that further testing can ensure the product has few faults or bugs. Sometimes, beta version are mde available to the open public to increase the feedback field to a maximal number of future users. 

Regression Testing:

after modifying the software, either for a change in functionality or the fix defects, a regression test re0uns previously passing tests on the modified 
software to ensure that the modifications haven't unintentionally caused a regression of prvious functionality. Regression testing can be performed at any or all of the 
above test levels. These regression tests are often automated. 

Sanity test: 

Sanity test is a basic test to quickly evaluate the validity of a claim or calculation. in mathematics, for example, when dividing by three or nine, verifying that the sum of the digits of the result is 
a multiple of 3 or 9 (casting out nines) respectively is a sanity test. 

Smoke testing; 
Smoke testing is a term used in plumbing, woodwind repair, electronics and computer software development. It refers to the first test made after repairs or first assembly to 
provide some assurance that the system under tests will catastrophically fail. After a smoke test proves that the pepes will not leak, the keys seal properly, the circuit will 
not burn, or the software wil not crash outright, the assemble is ready for more stressful testing. 


